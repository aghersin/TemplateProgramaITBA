\documentclass[a4paper,12pt]{article}
\usepackage{setspace}
\usepackage{url}
% \usepackage[a4paper,total={7.5in,25cm},top=2cm,bottom=2cm,headsep=0.3cm,footskip=0.5cm]{geometry}
% \usepackage[a4paper,total={6.5in,22cm},hoffset=1cm,footskip=1cm,headsep=2cm,voffset=1cm]{geometry}
\usepackage[a4paper,total={7in,24cm},hoffset=0cm,footskip=1cm,headsep=0cm,voffset=1cm,headheight=83pt]{geometry}

% \doublespacing
\singlespacing
% \onehalfspacing

% \usepackage[scaled]{uarial}
%\renewcommand*\familydefault{\sfdefault} 
%% Only if the base font of the document is to be sans serif

\usepackage[spanish]{babel}
\usepackage{floatflt}
\usepackage[utf8]{inputenc} % Required for inputting international characters
\usepackage[T1]{fontenc} % Use 8-bit encoding
\usepackage{csquotes}
\usepackage[backend=biber, style=apa]{biblatex}
\addbibresource{fundamentos.bib}
\DeclareBibliographyCategory{obligatoria}
\DeclareBibliographyCategory{complementaria}
\addtocategory{obligatoria}{paganini:rob-sys,rssp:rob-sys}
\addtocategory{complementaria}{doy:feed-cont}


%\usepackage{fourier} % Use the Adobe Utopia font for the document
%\usepackage{times}
\usepackage{helvet}

\renewcommand{\familydefault}{\sfdefault}
% If you use XeLaTeX or LuaLaTeX, you can use TrueType Arial font installed in your Windows/Mac:
% \usepackage{fontspec}
% \setmainfont{Arial}

\usepackage[pdftex]{graphicx}

\usepackage{fancyhdr}

\title{\underline{\LARGE DOCTORADO} \\ {\normalsize Programa analítico de la materia:}}
\author{}
\date{}
%\usepackage[pagestyles]{titlesec}
\usepackage{titlesec}

\usepackage{enumitem}
\setlist[enumerate]{itemsep=0mm}
\setlist[itemize]{itemsep=0mm}
\setlist[description]{itemsep=0mm}
\usepackage{xcolor}
\definecolor{lightgray}{rgb}{0.9, 0.9, 0.9}

%\titleformat*{\section}{\normalsize\normalfont\bfseries}
\titleformat{name=\section}[block]
  {\normalsize\normalfont\bfseries}%\sffamily\large}
  {}
  {0pt}
  {\colorsection}
\titlespacing*{\section}{0pt}{\baselineskip}{\baselineskip}

\newcommand{\colorsection}[1]{%
  \colorbox{lightgray}{\parbox{\dimexpr\textwidth-2\fboxsep}{\thesection\ #1}}}

\usepackage[colorlinks = true,
            linkcolor = blue,
            urlcolor  = blue,
            citecolor = blue,
            anchorcolor = blue]{hyperref}

\begin{document}

\fancyhead{} % clear all header fields
\renewcommand{\headrulewidth}{0pt}
\fancyhead[R]{\includegraphics*[width=40mm]{LogoITBA.jpg}}
\fancyfoot[L]{
\footnotesize 
Iguazú 341 - 2° piso \\
Parque Patricios, CABA \\
\url{https://www.itba.edu.ar}
} % clear all footer fields
\fancyfoot[C]{\thepage}
% \fancyfoot[LO,CE]{From: K. Grant}
% \fancyfoot[CO,RE]{To: Dean A. Smith}
\maketitle
\thispagestyle{fancy}
\pagestyle{fancy}
%\colorbox{lightgray}{
\section{{Nombre de la materia:} }

<AQUÍ VA EL NOMBRE DEL CURSO>.

\section{Presentación de la materia:\\{\tiny{\normalfont Hasta 500 palabras}}}

<AQUÍ VA LA PRESENTACIÓN DE LA MATERIA>.

\section{Docente responsable (Adjuntar CVar):}
\begin{enumerate}[label=\alph*-]
    \item Nombre y Apellido: <NOMBRE DEL DOCENTE>
    \item Máximo título alcanzado: <MAESTRÍA/DOCTORADO correspondiente.>.
    \item N° de DNI: xx.yyy.zzz
\end{enumerate}
\section{Equipo docente:}
<NOMBRE DEL DOCENTE 1>, <NOMBRE DEL DOCENTE 2>.
\section{
Requisitos de admisibilidad a la materia: \\
{\tiny{\normalfont (Cursos, técnicas o conocimientos previos necesarios para el cursado)}}}

<CONOCIMIENTOS PREVIOS>

\section{Duración en horas:}
\begin{enumerate}[label=\alph*-]
    \item Horas teóricas: 48 
    \item Horas prácticas: 16
    \item Horas total: 64
\end{enumerate}

\section{Idioma del dictado:}

<IDIOMA normalmente Castellano>.

\section{¿Podría dictarse una versión en idioma inglés?}

<SI o NO>.

\section{
Objetivos de aprendizaje: \\
{\tiny{\normalfont  Hasta tres.
}}}

\begin{enumerate}[label=\arabic*-]
    \item <OBJ1>.
    \item <OBJ2>.
    \item <OBJ3>.
\end{enumerate}

\section{ Contenidos \\ {{\normalfont \tiny Separado por unidades. }}}

Se cita como muestra 1 el libro de \cite{rssp:rob-sys}.

Se citan como muestra 2, los libros de \cite{paganini:rob-sys,doy:feed-cont}

\section{ Trabajo de laboratorio: \\ {{\normalfont \tiny De existir, se deberá consignar el tipo de actividad esperada. }}}

<DESARROLLAR EN BASE A SU CURSO>.
    
\section{ Metodología de enseñanza: \\ {{\normalfont \tiny Si se realiza alguna activ. práctica que no sea de laboratorio debe estar explicada en este punto.}}}

<MUESTRA DE METODOLOGÍA>

El curso se desarrollará en base a clases teórico–prácticas sobre la base de las cuales se asignará la realización de los siguientes trabajos prácticos:
\begin{enumerate}
\item Planteo de problemas de análisis y diseño.
\item Trabajo final.
\end{enumerate}

\section{ Bibliografía obligatoria: \\ {{\normalfont \tiny Referir según normas APA.}}}

\printbibliography[category={obligatoria},heading=none]

\section{ Bibliografía complementaria: \\ {{\normalfont \tiny Referir según normas APA.}}}

\printbibliography[category={complementaria},heading=none]

\section{ Recursos didácticos para la enseñanza:}

Aula, pizarrón, computadoras personales. <ADAPTAR A SU PREFERENCIA>

\section{ Modalidad de evaluación: \\ {{\normalfont \tiny Tipo, requerimiento, oportunidad, etc..}}}

Examen. <ADAPTAR A SU PREFERENCIA>

Trabajo práctico final. <ADAPTAR A SU PREFERENCIA>

\section{ Requisitos de aprobación: \\ {{\normalfont \tiny Condición, escala, etc.}}}

Nota mayor a seis para doctorado. <ADAPTAR A SU PREFERENCIA>

\end{document}
